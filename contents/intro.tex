% !TEX root = ../main.tex

\chapter{绪论}
近些年来,由于互联网的普及,计算机技术的发展和各种社交网络的兴起,视频和图片数据以爆炸性的速度快速增长。如何对海量的图像数据进行处理以及分析促进了一系列的计算机视觉技术的飞速发展,图像检索技术是其中一个核心研究方向。早期基于文本的图像检索极大依赖于标注者的主观判断的标签,检索效率以及性能低下同时需要耗费大量的人力资源,难以应用于海量的互联网图像数据检索。基于内容的的图像检索(Content Based Image Retrieval, CBIR)通过算法自动提取图像的视觉特征来衡量图片之间的相似度,从而实现在数据库中进行图像检索的目的,极大的节省了人工标注耗费的人力、物力和财力。为了提高检索速度以及降低存储开销,哈希编码以及量化编码方法被广泛应用到了图像检索领域,成为解决超大规模图像检索问题的重要解决方案。 本章重点介绍大规模图像检索算法的研究背景以及意义,概要的介绍了图像检索算法发展历史, 详细介绍了基于深度哈希的大规模的图像检索算法。 本章最后介绍了这篇论文的章节安排以及概括介绍每个章节包含的主要内容。
\section{本研究的背景及意义}
在大数据、移动互联网、计算机技术、传感器技术、云计算等前沿技术的发展以及经济社会急速发展的需求下,人工智能(artificial intelligence)作为被视为驱动新一轮产业变革的技术受到了世界各国的广泛关注。 广义的人工智能\cite{russell2010artificial} 包含对大数据的处理分析,自动推理能力,学习能力,以及解决问题的能力。而对以往存储的大数据信息的检索是后续学习与分析的基石,而图像检索便是其中关键的一部分。 \par
基于内容图像检索自从1990年代以来便受到了计算机视觉研究人员广泛的探索~\cite{}。如图所示, 典型的图像检索一般是指图片中进行特征提取,完成从原始图像空间到特征空间的映射,同时图片的特征应当保留原始图片的语义一致性。语义相似的图片的特征应当尽量接近,相反,语义不同的图片特征应该尽量远离。早期基于人工特征如 (SIFT, GIST, etc ) 的图像检索方法一般面临 ``语义鸿沟'' \cite*{bibid}的难题, 基于人工提取构建的低层次特征难以真实描绘保存高层次的语义特征。2012年, Krizhevsky 等人提出 ALexNet~\cite{} 并且在大规模视觉识别挑战赛中以比第二名低 $10.8\%$ 的 top-5 错误率的性能取得了冠军。自此,基于深度学习进行端对端的特征学习成为了计算机视觉领域的主流。通过设计不同的卷积神经网络(Convolutional Neural Network, CNN)结构,以及采用不同度量学习损失函数,基于深度学习的图像检索方法取得了引人注目的精度提升。自2021年来, Google 首次将自然语言处理领域的基于自注意力机制 (self-attention mechanism)的深度学习模型 transformer 应用到了计算机视觉领域, 并且在多个任务达到最先进水平。由于视觉 Transformer 具有较低的归纳偏置 (inductive bias), 在具有大规模监督数据的领域,它取得了比 CNN 更优异的图像表征学习能力。 因此探究其在图像检索领域的应用,也成为了当前的研究热点之一。  \par
另一方面,在图像领域的实际应用中。由于现在自媒体社交网络、

\section{国内外发展历史和研究现状}
\subsection{基于手工特征的图像检索}
\textbf
\subsection{基于深度学习的主干网络发展}
\subsection{基于深度特征的图像检索}
\subsection{基于深度哈希的大规模检索}

\section{本文章节安排与研究内容}