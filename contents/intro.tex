% !TEX root = ../main.tex

\chapter{绪论}
近些年来,由于互联网的普及,计算机技术的发展和各种社交网络的兴起,视频和图片数据以爆炸性的速度快速增长。如何对海量的图像数据进行处理以及分析促进了一系列的计算机视觉技术的飞速发展,图像检索技术是其中一个核心研究方向。早期基于文本的图像检索极大依赖于标注者的主观判断的标签,检索效率以及性能低下同时需要耗费大量的人力资源,难以应用于海量的互联网图像数据检索。基于内容的的图像检索(Content Based Image Retrieval, CBIR)通过算法自动提取图像的视觉特征来衡量图片之间的相似度,从而实现在数据库中进行图像检索的目的,极大的节省了人工标注耗费的人力、物力和财力。为了提高检索速度以及降低存储开销,哈希编码以及量化编码方法被广泛应用到了图像检索领域,成为解决超大规模图像检索问题的重要解决方案。 本章重点介绍大规模图像检索算法的研究背景以及意义,概要的介绍了图像检索算法发展历史, 详细介绍了基于深度哈希的大规模的图像检索算法。 本章最后介绍了这篇论文的章节安排以及概括介绍每个章节包含的主要内容。
\section{本研究的背景及意义}
在大数据、移动互联网、计算机技术、传感器技术、云计算等前沿技术的发展以及经济社会急速发展的需求下,人工智能(artificial intelligence)作为被视为驱动新一轮产业变革的技术受到了世界各国的广泛关注。 广义的人工智能\cite{russell2010artificial}
\section{国内外发展历史和研究现状}g
\subsection{基于手工特征的图像检索}
\subsection{基于深度学习的主干网络发展}
\subsection{基于深度特征的图像检索}
\subsection{基于深度哈希的大规模检索}
\subsubsection{最近邻搜索}
\subsubsection{近似最近邻搜索}
\subsubsection{基于深度哈希编码搜索}
\subsubsection{基于深度学习的量化编码搜索}
\section{本文章节安排与研究内容}