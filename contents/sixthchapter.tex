% !TEX root = ../main.tex
\chapter{总结与展望}
\section{本文工作总结}
随着互联网技术, 数字成像技术, 多媒体技术的蓬勃发展, 以及各种社交网络的兴起, 网络上的数字图像内容数量呈现爆炸性的速度的增长。如何对海量的图片进行处理与分析成为了现代社会一个急切的需求。大规模图像检索算法是对海量图片进行处理分析的基石, 由于其在智能安防, 电商搜图, 智能医疗等多个领域具有重要的应用价值, 图像检索技术在近年来获得了工业界以及学术界广泛的关注。然而针对真实的应用场景比如行人检索, 以及大规模的图像检索问题, 仍然是现在技术研究的主要瓶颈。 \par
针对上述的问题, 受到近年来深度哈希技术研究的启发, 本文主要研究了深度哈希技术在大规模图像检索领域的应用。我们针对两种图像检索的现实应用-行人检索-以及车辆检索进行探索。 同时, 受Vision Transformer在图像识别领域超越卷积神经网络的启发,我们探究了基于Vision Transformer的深度哈希技术在通用大规模图像检索的应用, 并且提出了两个新型的框架创新。本文提出的基于Vision Transformer的深度哈希框架是深度哈希领域第一个不基于卷积神经网络的框架, 具有理论意义以及较为显著的应用价值。 本文的主要创新点以及成果如下:\par
(1) 基于成对监督的细粒度跨模态图像检索 \par
传统行人重识别检索问题针对单模态数据, 由于模态鸿沟的存在, 无法被直接应用到跨模态行人重识别检索的场景下。 针对这一问题, 本文第二章提出了一个跨模态行人重识别检索的框架-\textbf{MAENET}。我们创新性的基于自编码器提出了一个神经网络结构来学习对不同模态恒定以及外表恒定的特征。 同时, 我们设计了适用于跨模态检索对齐不同模态特征的损失函数, 使得网络可以在动态创建的图片对上进行监督训练。第二章提出的框架在多个跨模态检索数据集上取得了优越的性能表现, 证明了算法的有效性。\par 
(2) 基于困难三元组的离散哈希检索 \par
传统的车辆重识别检索方法基于实值向量, 虽然可以取得较高的检索精度, 但是由于其存储开销大和检索速度慢, 无法适应真实环境的大规模车辆检索场景中。 本文第三章提出了第一个基于深度哈希的大规模车辆检索框架。通过提出了一个新型的离散哈希模块来进行离散的哈希码生成, 以及一个基于困难三元组的损失函数进行特征学习。本文第三章提出一个交替优化的算法来进行整个框架的优化。在主流的车辆重识别数据集上进行了四种不同哈希码长度的性能测试, 第三章提出的算法都显著超越了当前的普适哈希算法。 \par
(3) 基于自注意力机制的通用大规模检索 \par
为了进一步优化深度哈希算法的表征能力, 本文第四章第一个探索了不基于卷积神经网络的深度哈希框架。第四章提出了第一个完全基于Vision Transformer 的深度哈希算法(\textbf{TransHash})来进行大规模的图像检索。 我们设计了一个孪生Vision Transformer架构以及一个新型的双流特征学习的模块来进行细粒度的表征学习。 同时, 我们采取成对的基于贝叶斯的学习框架进行相似度保留的度量学习。 多个大规模通用图像检索的数据集上的实验结果表明了该框架相比较基于卷积神经网络的算法可以大幅度提升检索的性能。\par
(4) 基于自注意力机制的乘积量化图像检索算法 \par
由于基于传统汉明哈希编码的方法会带来较大的精度损失, 本文第五章探索了一种基于乘积量化编码的深度哈希算法来进一步提高检索精度。本文第一个提出基于基于Vision Transformer的乘积量化框架 (\textbf{APQFormer})。我们设计了一个基于双支Vision Transformer的乘积量化网络来进行多尺度细粒度的端到端乘积量化编码学习。同时设计了一个基于直接优化平均查准率的量化损失函数来进行度量学习。 我们在大规模通用检索数据集上进行了实验, 结果表明了该框架可以取得大幅度的性能提升。

\section{未来工作展望}
本文主要研究了大规模图像检索问题, 详细的探讨了如何将深度哈希技术应用到大规模图像检索的场景中。同时, 本文详细探究了基于自注意力机制的Vision Transformer在深度哈希中的应用, 并且针对不同的应用设计了创新性的网络架构, 取得了较大的性能提升。 本文研究的课题方向在未来还可以在下列几个场景下进行探索:
\begin{enumerate}
    \item 传统基于监督的图像检索方法需要耗费大量的人力物力进行图像的标注工作, 从而获得有标签的数据集进行监督训练。为了减少对监督标签的依赖, 无监督的图像检索方法已经获得了广泛的关注。 近年来对比学习(Contrastive Learning)在无监督学习领域取得了较大的成功。如何使用对比学习来进行无监督的深度哈希学习是一个值得探究的方向。
    \item 现阶段的深度哈希算法是基于平衡的数据集进行训练。然而在现实场景中, 图像的数据一般是呈现长尾分布(Long-tailed Distribution)。 针对长尾分布的数据进行哈希检索需要额外设计特定的网络结构以及特定的哈希损失函数, 是一个具有挑战性的研究方向。
    \item 基于深度哈希的视频检索是另外的一个热门的研究方向。 本文提出的基于Vision Transformer的架构思想可以被推广到大规模视频检索领域, 成为一个标准的基本主干网络模型。我们可以探究基于Vision Transformer的汉明哈希以及乘积量化方法来提高视频检索性能, 这是一个有前景的未来研究课题。
    \item 由于域差距(Domain Gap)的存在, 在源域上训练的算法在目标域数据集上进行测试时会带来较大的性能损失。为了减少重训练的成本, 实现一个更通用的检索模型。基于域自适应的深度哈希算法值得下一步进行仔细探索。结合扩散模型拟合数据的优势, 我们可以借助扩散模型(Diffusion Model)生成目标域的数据, 减少域差距带来的影响。这是下阶段值得研究的方向。
\end{enumerate}